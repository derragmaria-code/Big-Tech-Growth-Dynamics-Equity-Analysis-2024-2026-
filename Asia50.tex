\documentclass[11pt,a4paper]{article}
\usepackage[utf8]{inputenc}
\usepackage[T1]{fontenc}
\usepackage{amsmath, amsfonts, amssymb}
\usepackage{graphicx, booktabs, xcolor, hyperref, geometry, mdframed}

\geometry{margin=1in}
\definecolor{darkred}{RGB}{139,0,0}
\definecolor{navyblue}{RGB}{0,0,128}
\definecolor{lightgrey}{RGB}{245,245,245}

\title{
    \vspace{-0.5in}
    \textbf{\textcolor{darkred}{Geographic Factor Divergence:}} \\
    \large \textit{Growth-Decoupling in the US vs. Fundamental Alpha in Asia (2026)}
}
\author{\textbf{DERRAG Mariya} \\ \small Quantitative Research Portfolio}
\date{February 8, 2026}

\begin{document}

\maketitle

\section{Hypothesis \& Abstract}
This study performs a cross-border stress test on the \textbf{Growth-Dominant Regime} theory. By comparing the Top 20 US Tech constituents against the Top 50 Asian Market Leaders, we identify a critical geographic bias. While the US market has decoupled from revenue velocity ($P = 0.84$), the Asian market exhibits a robust, statistically significant correlation between growth and returns ($P < 0.0001$).

\section{The Asian Market "Clean Signal"}
In sharp contrast to the US "noise," the Asian dataset ($N=50$) demonstrates that fundamentals still dictate price action in the East.

\begin{mdframed}[backgroundcolor=lightgrey, linecolor=darkred, linewidth=1pt]
\centering
\textbf{Regional Statistical Verdict (Asia)} \\
Pearson Correlation ($r$): \textbf{0.6067} \quad | \quad P-Value: \textbf{0.0000} \\
\textit{Conclusion: Highly Significant. Revenue growth remains a primary Alpha factor in Asian equities.}
\end{mdframed}

\section{Comparative Data Analysis}
The Asian leaders, specifically in the semiconductor and hardware sectors, show a clear linear progression: high growth results in high returns.

\begin{table}[h!]
\centering
\caption{Top Asian Performance Tier (Feb 2025 -- Feb 2026)}
\label{tab:asia_results}
\begin{tabular}{@{}lccc@{}}
\toprule
\textbf{Ticker} & \textbf{Region} & \textbf{Revenue Growth} & \textbf{Total Return} \\ \midrule
\textbf{000660.KS} (SK Hynix) & Korea  & $+66.1\%$ & $+316.5\%$ \\
\textbf{005930.KS} (Samsung)  & Korea  & $+23.8\%$ & $+200.0\%$ \\
\textbf{2308.TW} (Delta)      & Taiwan & $+34.0\%$ & $+183.1\%$ \\
\textbf{2330.TW} (TSMC)       & Taiwan & $+20.5\%$ & $+62.2\%$ \\ \bottomrule
\end{tabular}
\end{table}

\section{The Core Divergence: US vs. Asia}
The primary findings suggest a "Regime Gap" driven by the following factors:
\begin{enumerate}
    \item \textbf{Valuation Maturity:} US tech (e.g., PLTR, META) has already priced in future growth, leading to "Growth Exhaustion." 
    \item \textbf{Execution vs. Narrative:} Asian returns are tied to physical production (Foundries/Memory). As demand for AI hardware remains high, firms like \textbf{SK Hynix} and \textbf{Samsung} are seeing direct fundamental-to-price translation.
    \item \textbf{The Efficiency Gap:} The US market is currently driven by \textit{narrative pivots} (e.g., Intel), whereas Asia remains a \textit{fundamentalist's market}.
\end{enumerate}



\section{Conclusion for Institutional Strategy}
The "Growth Paradox" is a \textbf{US-specific phenomenon}. For the remainder of 2026, quantitative strategies should favor:
\begin{itemize}
    \item \textbf{Short-Growth / Long-Strategic} in US Tech.
    \item \textbf{Long-Growth / Fundamental Tracking} in Asian Semis and Hardware.
\end{itemize}

\end{document}